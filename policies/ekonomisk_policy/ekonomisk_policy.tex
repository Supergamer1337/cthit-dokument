\documentclass[11pt, includeaddress]{classes/cthit}
\usepackage{titlesec}
\usepackage{verbatimbox}

\titleformat{\paragraph}[hang]{\normalfont\normalsize\bfseries}{\theparagraph}{1em}{}
\titlespacing*{\paragraph}{0pt}{3.25ex plus 1ex minus 0.2ex}{0.7em}

\graphicspath{ {images/} }

\begin{document}

\title{Ekonomisk policy}
\approved{2012--12--06}
\revisioned{2024--05--03}
\maketitle

\thispagestyle{empty}

\newpage

\makeheadfoot%

%Rubriksnivådjup
\setcounter{tocdepth}{2}
%Sidnumreringsstart
\setcounter{page}{1}
\tableofcontents

\newpage

\section{Syfte}
Denna policy syftar till att underlätta och ge kontinuitet i det ekonomiska arbetet på sektionen genom att samla riktlinjer för hur ekonomiska resurser bör användas.

\section{Anmärkningar}
\begin{itemize}
	\item Anmärkning 1: Alla siffror som avser svenska kronor i detta dokument är beräknade inklusive moms á 12,5\% för mat/dryck och 25\% för annat.
	\item Anmärkning 2: Detta dokument rör sektionens kommittéer
          samt sektionsstyrelsen \STYRIT{} och studienämnden \SNIT{}
          om inte annat nämns.
	\item Anmärkning 3: Om anledning finns att frångå policyn kan godkännande fås av \STYRIT{}.
	\item Anmärkning 4: Ett godkännande från styrelsen lämnas skriftligen från styrITs kassör som skall föregås av styrelsebeslut.
	\item Anmärkning 5: \SNIT{}s verksamhet som bekostats och godkänts av programledningen får avvika från den ekonomiska policyn tills annat beslut fattats av styrelsen.
\end{itemize}

\section{Slarv}
Slarv avser alla oförutsedda utgifter i form av böter, kontrollavgifter etc. som tillkommer i samband med kommitténs verksamhet. \\
Slarv betalas av antingen kommitté eller privatperson efter beslut av styrelsen. I detta beslut tas i beaktande huruvida händelsen bedöms ha inträffat på grund av vårdslöshet eller misstag. Har samma kommitté eller person gjort sig skyldig till flera överträdelser tas även detta i beaktning vid framtida bedömningar.

\section{Alkohol}
Sektionens pengar får inte gå till att bjuda på alkoholdryck (över 2.25 volymprocent). Undantag kan göras vid särskilda arrangemang, med godkännande av styrelsen.

\section{Mat under arrangemang}
Arbetsmat avser mat och dryck som förtärs av arbetande i samband med sektionsorganets verksamhet. 
Beroende på arbetspassets längd får varje person bjudas på mat till ett värde av x kronor enligt nedanstående tabell.

\begin{tabular}{ l  c  c  c}
    \centering
    Timmar & [0-3] & (3-6] & (6-$\infty$) \\
    \hline
    Kronor & 0 & 75 & 150 \\
\end{tabular}
\vspace{\the\baselineskip}

    Denna summa får inte överskrida 900 kr per person och verksamhetsår.

    Summorna är baserade på vad som anses vara en rimlig
    måltidskostnad.


\section{Överlämning}
Kostnaden för överlämning ska ej överstiga 265kr per person, där person avser både de som blivit invalda och de som är sittande vid det sektionsmöte där kommittén har sitt inval. Om det kan motiveras så finns det möjlighet att bekosta kringkostnader, såsom lokalhyror och resekostnader, vilka skall i förhand godkännas av styrelsen. Den totala kostnaden för överlämningen får ej överstiga 4500kr.

Exempel: FooIT’16 är 4 invalda och FooIT’17 är 5 invalda. Detta innebär att överlämningen från 16 till 17 har en max kostnad på 265*9 = 2385kr. Om summan av de två åren hade varit 17 eller större så hade 4500kr varit den gällande summan, då totala kostnaden ej får överstiga 4500kr.

\section{Representation}

\subsection{Profilering}
Profilering innefattar kläder och andra personliga artiklar vars syfte är att representera sektionsorganet. 
Tygmärken som införskaffas för att bytas bort bekostas inte av sektionen. \\

Profilering som bekostas av sektionen måste kunna skiljas från privat egendom med hjälp av till exempel tryck med sektionsorganets logotyp. \\

Införskaffad profilering ska användas vid relevanta tillfällen inom organets verksamhet. \\

Alla sektionsorgan får lägga upp till 800 kr per person på profilering.

\subsection{Kommittéutstyrsel}
Följande kommittéer får, utöver vad som beskrivs i §7.1, budgetera för en kommittéutstyrsel per person. 
För detta finns ingen övre kostnadsgräns men styrelsens godkännande krävs om kostnaden överstiger 1000 kr per person. \\

\ARMIT{}s utstyrsel innefattar en kavaj och därtill hörande färg, tryck och brodering. \\

\FRITID{}s utstyrsel innefattar ett träningsställ bestående av ett överkroppsplagg och ett underkroppsplagg och därtill hörande färg, tryck och brodering. \\

\FLASHIT{}s, \NOLLKIT{}s, \PRIT{}s, \SEXIT{}s och \CLOUDLORDS{} utstyrslar innefattar ett överkroppsplagg med tryck, ett kommittéplagg (’ovve’) och därtill hörande färg, tryck och brodering.

\subsection{Intern representation}
Intern representation avser tillfälliga och kortvariga aktiviteter som sker inom kommittén och är direkt kopplade till verksamheten. Exempel på detta är teambuilding och liknande. \\

Under ett tillfälle under första halvan av sektionsorganets mandatperiod är det tillåtet att spendera upp till 300 kr/person och tillfälle för intern representation. Utöver detta tillfälle får intern representation inte överstiga 150 kr/person och tillfälle. \\

Om det kan motiveras så finns det möjlighet att lägga till kringkostnader, så som lokalhyror, resekostnader, biljettkostnader vid arrangemang etc. Utgifter av denna typ skall i förhand godkännas av styrelsen. \\

Den totala kostnaden för intern representation inom kommittén får inte överstiga 900 kr/person och mandatperiod. 

\subsection{Representation inom sektionen}
Representation inom sektionen avser representation med minst ett annat sektionsorgan. \\

Under ett tillfälle är det tillåtet att spendera upp till 300 kr/person och tillfälle för representation inom sektionen. \\

Den totala kostnaden för representation inom sektionen får inte överstiga 450 kr/person och mandatperiod.

\subsection{Extern representation}
Extern representation avser representation med parter utanför kommittéer på IT. Extern representation bör ej förekomma om det inte anses nödvändigt för att ett samarbete med annan part skall fortsätta. Ett exempel på extern representation är tackkalas för puffar eller phaddrar. Utgifter som dessa måste godkännas av styrelsen.

\subsection{Begränsningar}
Pengar för intern representation får inte kombineras med pengar för representation inom sektionen.


\section{Inventarielista}
Inventarier syftar på större inköp eller objekt som har ett speciellt kontinuerligt värde för verksamheten. Ett exempel i dagsläget kan vara \NOLLKIT{}s bollhav som oavsett finansiellt värde har ett stort sentimentalt värde för kommittén.

Samtliga kommittéer skall föra inventarielista över dess ägodelar. På denna lista skall finnas:

\begin{itemize}
	\item Beskrivning av artikeln
	\item Inköpsdatum
	\item Inköpspris
	\item Kopia på inköpskvitto
\end{itemize}


\section{Utomstående personal}
Utomstående personal kan t ex avse musiker. Sådan typ av personal får endast anlitas om denne innehar F-skattesedel.

\section{Äskningar}
En äskning är en förfrågan till styrelsen om finansiella medel för att utföra en aktivitet eller göra ett inköp till nytta för sektionen och dess medlemmar. Görs äskningen av en kommitté så skall utgiften vara av en sådan art som inte täcks in av kommitténs huvudsakliga verksamhet eller som utan finansiell hjälp ej är genomförbar. \\

Äskningar skall skickas via mail till styrelsen i enlighet med mall som finns tillgänglig på sektionens hemsida. Varje äskning bedöms individuellt av styrelsen enligt ovanstående ramar inom 10 läsdagar.

\section{Större inköp}
Större inköp avser köp av enskilt objekt som uppgår till ett belopp som överstiger 1500 kr. Dessa inköp skall godkännas av styrelsen.

\section{Bokföring}
Kassörer ska uppvisa bokföring för sektionens lekmannarevisorer inför alla ordinarie sektionsmöten. 
Om kassör, utan god anledning, inte lämnar in sin bokföring (så väl digital som fysisk) i tid eller om den bokföring som lämnats in ej enligt lekmannarevisorerna håller tillräckligt hög kvalitet får berörd kommitté spenderings- och arrangemangsstopp fram tills dess att bokföring återigen uppvisats för och godkänts av lekmannarevisorerna.
Avgången kassör ska uppvisa sin bokföring för lekmannarrevisorer varje läsperiod tills att denne blivit ansvarsbefriad.

\section{Överenskomna övriga kostnader}
\begin{itemize}
	\item Samtliga medlemmar i \FANBARERIT{} har rätt att få sektionens frackband och pin bekostade av sektionen.
	\item Samtliga medlemmar i \FANBARERIT{} har, efter att ha genomfört minst ett arrangemang, rätt att var för sig få en kemtvätt av sin högtidsklädsel bekostad vid \FANBARERIT{}s verksamhetsårs slut.
        \item Samtliga arbetsgrupper har rätt till att lägga 800 kr per person på
intern representation, arbetsmat eller profilering.
\end{itemize}

\end{document}
